\documentclass{article}
\usepackage[utf8]{inputenc}
\usepackage[spanish]{babel}
\usepackage{amsmath}
\usepackage{amssymb}
\usepackage{amsfonts}
\usepackage{hyperref}
\usepackage{textcomp}
\usepackage{graphicx}
\usepackage{pgfplots}
\usepackage{geometry}
\usepackage{booktabs}
\hypersetup{
    colorlinks=true,
    linkcolor=black,
    citecolor=green,
    filecolor=magenta,      
    urlcolor=cyan,
}
\geometry{
  top=3cm,
  bottom=3cm,
  left=3cm,
  right=3cm
}

\title{Estadística 1}
\author{Jorge Miguel Alvarado Reyes}
\date{16 Agosto 2023}

\setlength{\parindent}{0pt}
\begin{document}

\begin{titlepage}
    \begin{center}
        \includegraphics[width=0.2\textwidth]{../../unam.png}
        \vspace*{.5cm}

        \LARGE
        \textbf{Universidad Nacional Autónoma de México}

        \vspace{0.5cm}
        \LARGE
        Facultad de Estudios Superiores Acatlán

        \vspace{2cm}

        \textbf{Estadística 2} \\
        Tarea unidad 1

        \vfill

        \vspace{1cm}

        \textbf{\large Autor:} \\
        Jorge Miguel Alvarado Reyes \\
        \vspace{.5cm}
        \normalsize \today

    \end{center}
\end{titlepage}
\newpage

\tableofcontents

\newpage

\section{Problema 1}
En algunas pruebas de salud en ancianos, un nuevo medicamento ha restaurado su memoria casi como la de jóvenes. Pronto se probará en pacientes con enfermedad de Alzheimer, esa fatal enfermedad del cerebro que destruye la mente. Según el Dr. Gary Lynch, de la Universidad de California en Irvine, el medicamento, llamado ampakina CX-516, acelera señales entre células cerebrales que parecen agudizar significativamente la memoria. 2En una prueba preliminar en estudiantes de poco más de 20 años y en hombres de entre 65 y 70 años de edad, los resultados fueron particularmente sorprendentes. Después de recibir dosis moderadas de este medicamento, las personas de entre 65 y 70 años de edad calificaron casi tan alto como los jóvenes. Los datos siguientes son los números de sílabas sin sentido recordadas después de 5 minutos, para 10 hombres de poco más de 20 años de edad y 10 señores de entre 65 y 70 años. Determina si las distribuciones para el número de sílabas sin sentido recordadas son iguales para estos dos grupos.

\begin{center}
    \begin{tabular}{c|c}
        20s    & 3 6 4 8 7 1 1 2 7 8 \\
        \hline
        65-70s & 1 0 4 1 2 5 0 2 2 3
    \end{tabular}
\end{center}

\begin{itemize}
    \item $H_0: F_x = F_y$ No hay diferencia entre las distribuciones de las cantidades de sílabas sin sentido recordadas por los dos grupos de edad.
    \item $H_1: F_x \neq F_y$ Existe una diferencia entre las distribuciones de las cantidades de sílabas sin sentido recordadas por los dos grupos de edad.
\end{itemize}

\subsection*{Ordenación de los Datos}

\[
    \{0, 0, 1, 1, 1, 1, 2, 2, 2, 2, 3, 3, 4, 4, 5, 6, 7, 7, 8, 8\}
\]

\subsection*{Cálculo de Rangos}

\begin{itemize}
    \item 0: 1.5
    \item 1: 4.5
    \item 2: 8.5
    \item 3: 11.5
    \item 4: 13.5
    \item 5: 15.0
    \item 6: 16.0
    \item 7: 17.5
    \item 8: 19.5
\end{itemize}

\subsection*{Suma de rangos para cada grupo}

\[ \text{20s: } 11.5 + 16 + 13.5 + 19.5 + 17.5 + 4.5 + 4.5 + 8.5 + 17.5 + 19.5 = 132.5 \]
\[ \text{65-70s: } 4.5 + 1.5 + 13.5 + 4.5 + 8.5 + 15 + 1.5 + 8.5 + 8.5 + 11.5 = 77.5\]

\[u = nm + \frac{n(n+1)}{2} - T_1\]
\[u_1 = 10 \cdot 10 + \frac{10(10+1)}{2} - 132.5 = 22.5\]
\[u_2 = 10 \cdot 10 + \frac{10(10+1)}{2} - 77.5 = 77.5\]

\[min(u_1, u_2) = 22.5\]

Calculando la \textbf{esperanza} tenemos:

\[E(u) = \frac{10 \cdot 10}{2} = 50\]

Calculando la \textbf{varianza} tenemos:

\[Var(u) = \frac{10 \cdot 10(10+10+1)}{12} = 39\]

\[z = \frac{u - E(u)}{\sqrt{Var(u)}} = \frac{22.5 - 50}{\sqrt{175}} = -2.0788\]

\[|z| = 2.0788\]

Realizando la prueba \( |z| > q_Z \left(1 - \frac{\alpha}{2} \right) \) obtenemos que.

\[
    q_Z (0.975) = 1.959964
\]

\[
    2.0788 > 1.959964
\]

De igual manera podemos hacer la prueba calculando el p-valor correspondiente.

\begin{align*}
    \text{p-valor} &= 2P(Z > 2.0788) \\
    &= 2 \left[1 - P(Z < 2.0788)\right] \\
    &= 2 \left[0.01881\right] \\
    &= 0.03763
\end{align*}

Sabemos que se rechaza $H_0$ (Se acepta $H_1$) si se cumple

\[\text{p-valor} < \alpha\]

\vspace*{.25cm}

En este caso la condicion 

\[0.03763 < 0.05\]

se cumple asi que concluimos que se rechaza $H_0$ asi que las distribuciones son diferentes

\section{Problema 2}

El tratamiento de cáncer por medios químicos, llamado quimioterapia, mata células cancerosas y células normales. En algunos casos, la toxicidad del medicamento para el cáncer, es decir, su efecto sobre células normales, puede reducirse con la inyección simultánea de un segundo medicamento. Se realizó un estudio para determinar si la inyección de un medicamento en particular reducía los efectos dañinos de un tratamiento de quimioterapia en el tiempo de sobrevivencia de ratas. Dos grupos de 12 ratas seleccionados al azar se emplearon en un experimento en el que ambos grupos, llamémoslos A y B, recibieron la droga tóxica en una dosis lo suficientemente grande para causarles la muerte, pero, además, el grupo B recibió la antitoxina que iba a reducir el efecto tóxico de la quimioterapia en células normales. La prueba finalizó al término de 20 días, o sea, 480 horas. Los tiempos de sobrevivencia para los dos grupos de ratas, a las 4 horas más cercanas, se muestran en la tabla siguiente. ¿Los datos dan suficiente evidencia para indicar que las ratas que recibieron la antitoxina tienden a sobrevivir más después de la quimioterapia que las que no recibieron la antitoxina?

\begin{center}
    \begin{tabular}{c c}
        Sólo quimioterapia & Quimioterapia más droga \\
        \hline
        84                 & 140                     \\
        128                & 184                     \\
        168                & 368                     \\
        92                 & 96                      \\
        184                & 480                     \\
        92                 & 188                     \\
        76                 & 480                     \\
        104                & 244                     \\
        72                 & 440                     \\
        180                & 380                     \\
        144                & 480                     \\
        120                & 196                     \\
    \end{tabular}
\end{center}

\section{Problema 3}

Dos chefs, A y B, calificaron 22 comidas en una escala del 1 al 10. Los datos se muestran en la tabla. ¿Los datos dan suficiente evidencia para indicar que uno de los chefs tiende a dar calificaciones más altas que el otro?
\begin{center}
    \begin{tabular}{c c c| c c c}
        Comida & A & B & Comida & A & B  \\
        \hline
        1      & 6 & 8 & 12     & 8 & 5  \\
        2      & 4 & 5 & 13     & 4 & 2  \\
        3      & 7 & 4 & 14     & 3 & 3  \\
        4      & 8 & 7 & 15     & 6 & 8  \\
        5      & 2 & 3 & 16     & 9 & 10 \\
        6      & 7 & 4 & 17     & 9 & 8  \\
        7      & 9 & 9 & 18     & 4 & 6  \\
        8      & 7 & 8 & 19     & 4 & 3  \\
        9      & 2 & 5 & 20     & 5 & 4  \\
        10     & 4 & 3 & 21     & 3 & 2  \\
        11     & 6 & 9 & 22     & 5 & 3  \\
    \end{tabular}
\end{center}

\section{Problema 4}
Dos métodos para controlar el tránsito, A y B, se usaron en cada una de n = 12 cruceros durante una semana y los números de accidentes que ocurrieron durante ese tiempo se registraron. El orden de uso (cuál se emplearía para la primera semana) se seleccionó de una manera aleatoria. Se desea saber si los datos dan suficiente evidencia para indicar una diferencia en las distribuciones de porcentajes de accidentes para los métodos A y B de control de tránsito.

\begin{center}
    \begin{tabular}{c c c |c c c}
        Crucero & A & B & Crucero & A & B \\
        \hline
        1       & 5 & 4 & 7       & 2 & 3 \\
        2       & 6 & 4 & 8       & 4 & 1 \\
        3       & 8 & 9 & 9       & 7 & 9 \\
        4       & 3 & 2 & 10      & 5 & 2 \\
        5       & 6 & 3 & 11      & 6 & 5 \\
        6       & 1 & 0 & 12      & 1 & 1 \\
    \end{tabular}
\end{center}

\subsection*{Solución}
Dado que tenemos datos en pares y lo que buscamos demostar es si la distribucion de los datos diferen podemos utilizar una prueba Wilcoxon

\begin{center}
    \begin{tabular}{c c c c c c}
        Linea A & Linea B & $ A - B $ & $| A - B |$ & Rango & R con signo \\
        \hline
        5       & 4       & $1$       & 1           & 3.5   & 3.5         \\
        6       & 4       & $2$       & 2           & 7.5   & 7.5         \\
        8       & 9       & $-1$      & 1           & 3.5   & -3.5        \\
        3       & 2       & $1$       & 1           & 3.5   & 3.5         \\
        6       & 3       & $3$       & 3           & 10    & 10          \\
        1       & 0       & $1$       & 1           & 3.5   & 3.5         \\
        2       & 3       & $-1$      & 1           & 3.5   & -3.5        \\
        4       & 1       & $3$       & 3           & 10    & 10          \\
        7       & 9       & $-2$      & 2           & 7.5   & -7.5        \\
        5       & 2       & $3$       & 3           & 10    & 10          \\
        6       & 5       & $1$       & 1           & 3.5   & 3.5         \\
        1       & 1       & $0$       & 0           & 0     & 0           \\
    \end{tabular}
\end{center}

\[T_{+} = 3.5 + 7.5 + 3.5 + 10 + 3.5 + 10 + 10 + 3.5 = 51.5\]
\[T_{-} = 3.5 + 3.5 + 7.5 = 14.5\]
\[T = min(T_{+}, T_{-}) = 14.5\]

\[
    E(T) = \frac{n(n + 1)}{4} = \frac{12(12 + 1)}{4} = 39
\]
\[
    Var(T) = \frac{n(n + 1)(2n + 1)}{24} = \frac{12(12 + 1)(2(12) + 1)}{24} =169
\]
\[
    z = \frac{T - E(T)}{\sqrt{Var(T)}} = \frac{17.5 - 39}{\sqrt{169}} = -1.65385
\]

Tomando $\alpha = 0.05$

\[q_z(1 - \frac{\alpha}{2}) = 1.96\]

Se rechaza $H_0$ si

\[|z| > q_z(1-\frac{\alpha}{2})\]

\[1.65385 > 1.96\]

Por lo anto las distribuciones son iguales (No se rechaza $H_0$), se acepta $H_0$ lo que indica que la lina A y la linea B no son diferentes

\section{Problema 5}
Los resultados de un experimento para investigar el reconocimiento de productos, durante tres campañas publicitarias, se muestran en la siguiente tabla. Las respuestas fueron el porcentaje de 400 adultos que estaban familiarizados con el producto recién anunciado. La gráfica de probabilidad normal indicó que los datos no eran aproximadamente normales y debía usarse otro método de análisis. ¿Hay una diferencia significativa entre las tres distribuciones poblacionales de donde vinieron estas muestras?

\begin{center}
    \begin{tabular}{c |c |c}
        \multicolumn{3}{c}{Campaña} \\
        \hline
        1   & 2   & 3               \\
        .33 & .28 & .21             \\
        .29 & .41 & .30             \\
        .21 & .34 & .26             \\
        .32 & .39 & .33             \\
        .25 & .27 & .31             \\
    \end{tabular}
\end{center}

\subsection*{Solución}

Como tenemos tres muestras, las purebas de Whitney y Wilcoxon quedan descartadas, usaremos una prueba de de Kruskal-Wallis

\textbf{Asignacion de rangos}

\begin{center}
    \begin{tabular}{c |c |c}
        \multicolumn{3}{c}{Campaña}              \\
        \hline
        1            & 2          & 3            \\
        $.33_{11.5}$ & $.28_{6}$  & $.21_{1.5}$  \\
        $.29_{7}$    & $.41_{15}$ & $.30_{8}$    \\
        $.21_{1.5}$  & $.34_{13}$ & $.26_{4}$    \\
        $.32_{10}$   & $.39_{14}$ & $.33_{11.5}$ \\
        $.25_{3}$    & $.27_{5}$  & $.31_{9}$    \\
    \end{tabular}
\end{center}

\begin{table}[ht]
    \centering
    \caption{\textbf{Número de observaciones y suma de rangos}}
    \begin{tabular}{cccc}
        \toprule
        1          & 2          & 3          \\
        \midrule
        $n_1 = 5$  & $n_2 = 5$  & $n_3 = 5$  \\
        $R_1 = 33$ & $R_2 = 53$ & $R_3 = 34$ \\
        \bottomrule
    \end{tabular}
\end{table}

\[ H = \frac{12}{15 \times 16} \left(\frac{33^2}{5} + \frac{53^2}{5} + \frac{34^2}{5} \right) - 3(15 + 1) = 2.5399 \]

Se rechaza $H_0$ si

\[H > q_{X^{2}_{k-1}}(1 - \alpha)\]
\[2.5399 > q_{X^{2}_{2}}(0.95) = 5.991\]

Se rechaza \(H_0\), no hay evidencia suficiente para afirmar que existen diferencias significativas entre las distribuciones poblacionales de las tres campañas publicitarias.Esto significa que, las tres campañas parecen ser igualmente efectivas.

\section{Problema 6}
En un estudio reciente que involucró una muestra aleatoria de 300 accidentes automovilísticos, se clasificó la información de acuerdo con el tamaño del automóvil.

\begin{center}
    \begin{tabular}{c| c c c}
                               & Pequeño & Mediano & Grande \\
        \hline
        Por lo menos un muerto & 42      & 35      & 20     \\
        Ningún muerto          & 78      & 65      & 60     \\
    \end{tabular}
\end{center}

Con estos datos,  ¿puedes afirmar que la frecuencia de accidentes depende del  tamaño del automóvil?

\begin{center}
    \begin{tabular}{c| c c c c}
                               & Pequeño & Mediano & Grande & Suman Filas \\
        \hline
        Por lo menos un muerto & 42      & 35      & 20     & 97          \\
        Ningún muerto          & 78      & 65      & 60     & 203         \\
        Suma Columnas          & 120     & 100     & 80     & 300
    \end{tabular}
\end{center}
\begin{center}
    \begin{tabular}{c| c c c}
                               & Pequeño & Mediano & Grande \\
        \hline
        Por lo menos un muerto & 38.8    & 32.3    & 26.867 \\
        Ningún muerto          & 81.2    & 67.67   & 54.13  \\
    \end{tabular}
\end{center}
\begin{align*}
     & \sum\limits_{i} 0.26391753 + 0.21993127 + 1.33058419 + 0.12610837 + 0.10509031 + 0.63579639 = 2.68142806
\end{align*}

\section{Problema 7}

Verifica si los siguientes datos provienen de una distribución normal.
\begin{center}
    \begin{tabular}{c c c c c}
        1.0672029  & 2.3103976  & 0.8193199  & -0.8588287 & 0.8003015  \\
        -0.8404432 & 0.2049356  & -0.9665391 & 1.7639849  & -0.7825124 \\
        -2.9712801 & -0.8181979 & -2.0191393 & -1.6289196 & 2.4613544  \\
        -2.6406738 & -2.6125324 & -1.1968322 & -0.1210923 & -2.1779296 \\
        2.5898699  & -2.6133718 & -1.6105999 & 1.0137149  & -2.3441204 \\
    \end{tabular}
\end{center}

\section{Problema 8}

Demuestra que el coeficiente de la $\tau$ de Kendall es:

$$\tau=1-\frac{4Q}{n(n-1)}$$

Hint: P+Q=$\frac{n(n-1)}{2}$

\section{Problema 9}

Investiga la prueba de Friedman y resuelve un problema práctico.

\section{Problema 10}

Investiga la prueba de independencia basada en el coeficiente de Pearson y resuelve un problema práctico.

\end{document}
