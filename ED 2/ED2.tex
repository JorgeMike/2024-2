\documentclass{article}
\usepackage[utf8]{inputenc}
\usepackage[spanish]{babel}
\usepackage{amsmath}
\usepackage{amssymb}
\usepackage{amsfonts}
\usepackage{hyperref}
\usepackage{textcomp}
\usepackage{graphicx}
\usepackage{pgfplots}
\usepackage{geometry}
\hypersetup{
    colorlinks=true,
    linkcolor=black,
    citecolor=green,
    filecolor=magenta,      
    urlcolor=cyan,
}
\geometry{
  top=3cm,            % Margen superior
  bottom=3cm,         % Margen inferior
  left=3cm,           % Margen izquierdo
  right=3cm           % Margen derecho
}

\title{Estadística 1}
\author{Jorge Miguel Alvarado Reyes}
\date{16 Agosto 2023}

\setlength{\parindent}{0pt}
\begin{document}

\begin{titlepage}
    \begin{center}
        \includegraphics[width=0.2\textwidth]{../unam.png}
        \vspace*{.5cm}

        \LARGE
        \textbf{Universidad Nacional Autónoma de México}

        \vspace{0.5cm}
        \LARGE
        Facultad de Estudios Superiores Acatlán

        \vspace{2cm}

        \textbf{Apuntes} \\
        Ecuaciones Diferenciales

        \vfill

        \vspace{1cm}

        \textbf{\large Autor:} \\
        Jorge Miguel Alvarado Reyes \\
        \vspace{.5cm}
        \normalsize \today

    \end{center}
\end{titlepage}
\newpage

\tableofcontents

\newpage

\section{30/01/2024}

\subsection{Contacto}

\begin{itemize}
    \item Email: Camo6812@acatlan.unam.mx
\end{itemize}

\subsection{Evaluación}

\begin{itemize}
    \item Tareas 60\% Examenes 40\%
\end{itemize}

\textbf{Tarea}

Tomarse una selfi con una de las referencias bibliograficas

\section{01/02/2024}

\subsection{Transformadas integrales}

Las transformadas integrales son operadores que asocian nuevas funciones a un determinado conjunto mediante integración respecto a un determinado parámetro. La forma general de una transformada integral se puede expresar como:

\[
    T\{f(t)\} = F(s) = \int_{a}^{b} K(s, t) f(t) \, dt
\]

Donde:
\begin{itemize}
    \item $T$ representa el operador de transformada.
    \item $f(t)$ es la función original que queremos transformar.
    \item $F(s)$ es la función transformada, que depende de la variable $s$.
    \item $K(s, t)$ es el núcleo de la transformada, una función que depende tanto de la variable original $t$ como de la nueva variable $s$.
    \item $a$ y $b$ son los límites de integración, que pueden ser finitos o infinitos dependiendo del tipo específico de transformada integral.
\end{itemize}

Las transformadas integrales más conocidas y utilizadas incluyen:

\begin{itemize}
    \item \textbf{La Transformada de Fourier}, donde $a = -\infty$, $b = \infty$, y $K(s, t) = e^{-2\pi ist}$, utilizada ampliamente en el análisis de señales y sistemas.
    \item \textbf{La Transformada de Laplace}, con $a = 0$, $b = \infty$, y $K(s, t) = e^{-st}$, muy usada en la solución de ecuaciones diferenciales y en la teoría de control.
    \item \textbf{La Transformada de Mellin}, que es otra forma de transformada integral con un núcleo específico que permite transformaciones de productos en convoluciones, útil en teoría de números y análisis complejo.
\end{itemize}

Estas transformadas convierten funciones del dominio del tiempo o espacio (dominio $t$) a funciones en el dominio de frecuencias o complejo (dominio $s$), facilitando la manipulación matemática y la solución de problemas complejos.

\vspace{.25cm}

\textbf{Tarea 2}

En \url{https://es.wikipedia.org/wiki/Transformada_integral} hay una tabla con las tranformaciones y sus inversas. Elabore una tabla similar pero con nuestra simbologia. Para cada tranformada intente calcular la de $f(t) = t + c$

\[
    \lim_{\substack{a \to -\infty \\ b \to \infty}} \int_{a}^{b} \frac{1}{\pi}\left(\frac{1}{s-c-t}\right) \, dt
\]

\[
    \lim_{\substack{a \to -\infty \\ b \to \infty}}  \frac{1}{\pi} \int_{a}^{b} \left(\frac{1}{s-c-t}\right) \, dt
\]

\[
    \lim_{\substack{a \to -\infty \\ b \to \infty}}  \frac{1}{\pi} \int_{a}^{b} \left(\frac{1}{u}\right) - \, du
\]

\[ \lim_{\substack{a \to -\infty \\ b \to \infty}}  \frac{1}{\pi} - \left[\ln(u)\right]_{a}^{b}\]

\textbf{Ejercicios}

La tranformada de Laplace y condiciones de existencia

Para calcular la transformada de Laplace de \(f(t) = \sin(t)\), utilizamos la definición:
\[
    \mathcal{L}\{\sin(t)\} = \int_{0}^{\infty} e^{-st} \sin(t) \, dt
\]

Aplicamos la técnica de integración por partes, donde elegimos:
\begin{align*}
    u  & = \sin(t),    & dv & = e^{-st} dt                            \\
    du & = \cos(t) dt, & v  & = \int e^{-st} dt = -\frac{1}{s}e^{-st}
\end{align*}

La fórmula de integración por partes \(\int u dv = uv - \int v du\) nos lleva a:
\[
    \int_{0}^{\infty} e^{-st} \sin(t) \, dt = \left[-\frac{1}{s} \sin(t) e^{-st}\right]_{0}^{\infty} + \frac{1}{s} \int_{0}^{\infty} e^{-st} \cos(t) \, dt
\]

\[
    \int_{0}^{\infty} e^{-st} \sin(t) \, dt = 0 + \frac{1}{s} \int_{0}^{\infty} e^{-st} \cos(t) \, dt
\]

Ahora debemos calcular

\[
    \frac{1}{s} \int_{0}^{\infty} e^{-st} \cos(t) \, dt
\]

Aplicamos la técnica de integración por partes, donde elegimos:
\begin{align*}
    u  & = \cos(t),     & dv & = e^{-st} dt                            \\
    du & = -\sin(t) dt, & v  & = \int e^{-st} dt = -\frac{1}{s}e^{-st}
\end{align*}

\[
    \frac{1}{s} \int_{0}^{\infty} e^{-st} \cos(t) \, dt = \left[-\frac{1}{s} \cos(t) e^{-st}\right]_{0}^{\infty} + \frac{1}{s} \int_{0}^{\infty} e^{-st} - \sin(t) \, dt
\]

\[
    \frac{1}{s}\int_{0}^{\infty} e^{-st} \cos(t) \, dt = 0 + \frac{1}{s} \int_{0}^{\infty} e^{-st} - \sin(t) \, dt
\]

\[
    \frac{1}{s}\int_{0}^{\infty} e^{-st} \cos(t) \, dt =\frac{1}{s} \left(- \frac{1}{s} \int_{0}^{\infty} e^{-st} \sin(t) \, dt\right)
\]

\[
    \frac{1}{s}\int_{0}^{\infty} e^{-st} \cos(t) \, dt =\left(-\frac{1}{s^2} \int_{0}^{\infty} e^{-st} \sin(t) \, dt\right)
\]

\newpage

Para calcular la transformada de Laplace de \(f(t) = \cos(\alpha t)\), utilizamos la definición:
\[
    \mathcal{L}\{\cos(\alpha t)\} = \int_{0}^{\infty} e^{-st} \cos(\alpha t) \, dt
\]

Aplicamos la técnica de integración por partes, donde elegimos:
\begin{align*}
    u  & = \cos(\alpha t),         & dv & = e^{-st} dt           \\
    du & = -\alpha \sin(\alpha t), & v  & =  -\frac{1}{s}e^{-st}
\end{align*}

La fórmula de integración por partes \(\int u dv = uv - \int v du\) nos lleva a:

\begin{align*}
    \int_{0}^{\infty} \cos(\alpha t) e^{-st} \, dt & = \left[\cos(\alpha t) -\frac{1}{s}e^{-st}\right]_{0}^{\infty} - \int_{0}^{\infty} -\frac{1}{s}e^{-st} - \alpha \sin(\alpha t) \, dt
\end{align*}

\[\left[\cos(\alpha t) -\frac{1}{s}e^{-st}\right]_{0}^{\infty} = \left[\cos(\alpha \infty) -\frac{1}{s}e^{-s\infty} - \cos(\alpha 0) -\frac{1}{s}e^{-s 0}\right]\]

\[\left[\cos(\alpha \infty) -\frac{1}{s}e^{-s\infty} - \cos(\alpha 0) -\frac{1}{s}e^{-s 0}\right] = \frac{1}{s}\]

\begin{align*}
    \int_{0}^{\infty} \cos(\alpha t) e^{-st} \, dt & = \frac{1}{s} - \frac{\alpha}{s} \int_{0}^{\infty} e^{-st}  \sin(\alpha t) \, dt
\end{align*}

Debemos resolver

\[ \int_{0}^{\infty} e^{-st} \sin(\alpha t) \, dt\]

Aplicamos la técnica de integración por partes, donde elegimos:
\begin{align*}
    u  & = \sin(\alpha t),        & dv & = e^{-st} dt           \\
    du & = \alpha \cos(\alpha t), & v  & =  -\frac{1}{s}e^{-st}
\end{align*}

\begin{align*}
    \int_{0}^{\infty} \sin(\alpha t) e^{-st} \, dt & = \left[ \sin(\alpha t) \cdot -\frac{1}{s}e^{-st}\right]_{0}^{\infty} - \int_{0}^{\infty} -\frac{1}{s}e^{-st} \cdot \alpha \cos(\alpha t)
\end{align*}

\[\left[ \sin(\alpha t) \cdot -\frac{1}{s}e^{-st}\right]_{0}^{\infty} = \left[\sin(\alpha \infty) \cdot -\frac{1}{s}e^{-s\infty} - \sin(\alpha 0) \cdot -\frac{1}{s}e^{-s0}\right] = 0\]

\begin{align*}
    \int_{0}^{\infty} \sin(\alpha t) e^{-st} \, dt & = 0 - \int_{0}^{\infty} -\frac{1}{s}e^{-st} \cdot \alpha \cos(\alpha t) \\
                                                   & = \frac{\alpha}{s} \int_{0}^{\infty} e^{-st} \cos(\alpha t)
\end{align*}

\section{08/02/2024}

\subsection{Teoremas de la tranformada de laplace}

\[
    \mathcal{L}\{c\} = \frac{c}{s}
\]

\[
    \mathcal{L}\{t^n\} = \frac{n!}{s^{n+1}}
\]

\[
    \mathcal{L}\{sen(\alpha t)\} = \frac{\alpha}{s^2 + \alpha^2}
\]

\[
    \mathcal{L}\{cos(\alpha t)\} = \frac{s}{s^2 + \alpha^2}
\]

\[
    \mathcal{L}\{e^{\alpha t}\} = \frac{1}{s - \alpha}
\]

\[
    \mathcal{L}\{senh(\alpha t)\} = \frac{\alpha}{s^2 - \alpha^2}
\]

\[
    \mathcal{L}\{cosh(\alpha t)\} = \frac{s}{s^2 - a^2}
\]

\newpage

\section{Tarea 4}

\begin{enumerate}
    \item $y' + 4y = e^{-4t}, \hspace{.5cm} y(0) = 2 \hspace{.5cm} y'(0)=0$
    \item $y' - y = 1 - te^{t}, \hspace{.5cm} y(0) = 0$
    \item $y'' + 2y' + y = 0, \hspace{.5cm} y(0) = y'(0) = 1$
    \item $y'' - 4y' + 4y = t^{3}e^{2t}, \hspace{.5cm} y(0) = y'(0) = 0 $
    \item $y'' - 6y' + 9y = t, \hspace{.5cm} y(0) = 0, y'(0) = 1$
    \item $y'' - 4y' + 4y = t^3, \hspace{.5cm} y(0) = 1, y'(0) = 0 $
    \item $y'' - 6y' + 13y = 0, \hspace{.5cm} y(0) = 0, y'(0) = -3$
    \item $2y'' + 20y' + 51y = 0, \hspace{.5cm} y(0) = 2, y'(0) = 0$
    \item $y'' - y' = e^{t}cos(t), \hspace{.5cm} y(0) = 0, y'(0) = 0 $
    \item $y'' - 2y' + 5y = 1 + t, \hspace{.5cm} y(0) = 0, y'(0) = 4$
\end{enumerate}

\newpage

\subsection{Solución}

\subsubsection{Problema 1}
\[y' + 4y = e^{-4t}, \hspace{.5cm} y(0) = 2\]

Aplicar la Transformada de Laplace en la ecuación es:
\begin{equation*}
    \mathcal{L}\{y'\} + 4\mathcal{L}\{y\} = \mathcal{L}\{e^{-4t}\}.
\end{equation*}

Resolviendo $\mathcal{L}\{e^{-4t}\}$:

\begin{align*}
    \mathcal{L}\{e^{-4t}\} & = \int_0^\infty e^{-st}e^{-4t}\,dt                  \\
                           & = \int_0^\infty e^{-(s+4)t}\,dt                     \\
                           & = \left[ -\frac{1}{s+4}e^{-(s+4)t} \right]_0^\infty \\
                           & = \frac{1}{s + 4}
\end{align*}

Resolviendo $\mathcal{L}\{y'\}$ y $\mathcal{L}\{y\}$
\begin{align*}
    \mathcal{L}\{y'\} & = sY(s) - y(0), \\
    \mathcal{L}\{y\}  & = Y(s),
\end{align*}

Sustituyendo con la condicion inicial \(y(0) = 2\) en la ecuación:
\begin{equation*}
    sY(s) - 2 + 4Y(s) = \frac{1}{s + 4}
\end{equation*}

Esto simplifica a:
\begin{equation*}
    Y(s)(s + 4) = 2 + \frac{1}{s + 4}
\end{equation*}

Despejamos \(Y(s)\):
\begin{equation*}
    Y(s) = \frac{2}{s + 4} + \frac{1}{(s + 4)^2}
\end{equation*}

\[
    Y(s) = \frac{2s+9}{(s + 4)^2}
\]

\newpage

\subsubsection{Problema 2}
\[y' - y = 1 - te^{t}, \hspace{.5cm} y(0) = 0\]
Aplicamos la Transformada de Laplace a ambos lados de la ecuación:
\[\mathcal{L}\{y'\} - \mathcal{L}\{y\} = \mathcal{L}\{1 - te^{t}\}\]

Utilizamos las siguientes propiedades de la Transformada de Laplace:
\[
    \mathcal{L}\{1\} = \frac{1}{s}, \quad \mathcal{L}\{te^{\alpha t}\} = \frac{1}{(s - \alpha)^2}
\]

Por lo tanto, para \(\mathcal{L}\{1 - te^{t}\}\) tenemos:
\[
    \mathcal{L}\{1\} - \mathcal{L}\{te^{t}\} = \frac{1}{s} - \frac{1}{(s-1)^2}
\]

Para \(\mathcal{L}\{y'\}\) y \(\mathcal{L}\{y\}\), considerando la condición inicial \(y(0) = 0\), obtenemos:
\[
    \mathcal{L}\{y'\} = sY(s) - y(0) = sY(s)
\]
\[
    \mathcal{L}\{y\} = Y(s)
\]

Sustituyendo en la ecuación transformada:
\[
    sY(s) - Y(s) = \frac{1}{s} - \frac{1}{(s-1)^2}
\]

Resolviendo para \(Y(s)\):
\[
    Y(s)(s - 1) = \frac{1}{s} - \frac{1}{(s-1)^2}
\]

\[
    Y(s) = \frac{1}{s(s - 1)} - \frac{1}{(s-1)^3}
\]

\newpage


\subsubsection{Problema 3}
\[y'' + 2y' + y = 0, \hspace{.5cm} y(0) = y'(0) = 1\]

Aplicamos la Transformada de Laplace a ambos lados de la ecuación:
\[\mathcal{L}\{y''\} + 2\mathcal{L}\{y'\} + \mathcal{L}\{y\} = \mathcal{L}\{0\}\]

Para \(\mathcal{L}\{y''\}\), \(\mathcal{L}\{y'\}\) y \(\mathcal{L}\{y\}\):

\begin{align*}
    \mathcal{L}\{y''\} & = s^2Y(s) - sy(0) - y'(0) \\
    \mathcal{L}\{y'\}  & = sY(s) - y(0)            \\
    \mathcal{L}\{y\}   & = Y(s)
\end{align*}

Considerando la condición inicial \(y(0) = y'(0) = 1\), tenemos:

\begin{align*}
    \mathcal{L}\{y''\} & = s^2Y(s) - s(1) - 1 \\
    \mathcal{L}\{y'\}  & = sY(s) - 1          \\
    \mathcal{L}\{y''\} & = Y(s)
\end{align*}

Sustituyendo:

\[
    s^2Y(s) - s - 1 + 2(sY(s) - 1) + Y(s) = 0
\]

\[
    s^2Y(s) - s - 1 + 2sY(s) - 2 + Y(s) = 0
\]

\[
    Y(s)(s^2 + 2s + 1) - s - 3 = 0
\]

\[
    (s^2 + 2s + 1)Y(s) = s + 3
\]

Despejando $Y(s)$

\[
    Y(s) = \frac{s + 3}{(s + 1)^2}
\]

\newpage


\subsubsection{Problema 4}
\[
    y'' - 4y' + 4y = t^{3}e^{2t}, \hspace{.5cm} y(0) = y'(0) = 0
\]

Aplicamos la Transformada de Laplace a ambos lados de la ecuación:
\[
    \mathcal{L}\{y''\} - 4\mathcal{L}\{y'\} + 4\mathcal{L}\{y\} = \mathcal{L}\{t^3e^{2t}\}.
\]

para \(\mathcal{L}\{t^3e^{2t}\}\) tenemos:

\begin{equation*}
    \mathcal{L}\{t^3e^{2t}\} = \frac{3!}{(s-2)^4} = \frac{6}{(s-2)^4}.
\end{equation*}

Para \(\mathcal{L}\{y''\}\), \(\mathcal{L}\{y'\}\) y \(\mathcal{L}\{y\}\):
\begin{align*}
    \mathcal{L}\{y''\} & = s^2Y(s) - sy(0) - y'(0) \\
    \mathcal{L}\{y'\}  & = sY(s) - y(0)            \\
    \mathcal{L}\{y\}   & = Y(s)
\end{align*}

Considerando la condición inicial \(y(0) = y'(0) = 0\), tenemos:
\begin{align*}
    \mathcal{L}\{y''\} & = s^2Y(s) - s(0) - 0 \\
    \mathcal{L}\{y'\}  & = sY(s) - 0          \\
    \mathcal{L}\{y\}   & = Y(s)
\end{align*}

Sustituyendo:
\begin{equation*}
    s^2Y(s) - 4sY(s) + 4Y(s) = \frac{6}{(s-2)^4}.
\end{equation*}

Factorizamos el término en \(Y(s)\) y simplificamos:
\begin{equation*}
    Y(s)(s^2 - 4s + 4) = \frac{6}{(s-2)^4},
\end{equation*}
\begin{equation*}
    Y(s)(s - 2)^2 = \frac{6}{(s-2)^4}.
\end{equation*}

Despejamos \(Y(s)\):
\begin{equation*}
    Y(s) = \frac{6}{(s-2)^6}.
\end{equation*}


\newpage


\subsubsection{Problema 5}

\[y'' - 6y' + 9y = t, \hspace{.5cm} y(0) = 0, y'(0) = 1\]

Aplicamos la Transformada de Laplace a ambos lados de la ecuación:
\[
    \mathcal{L}\{y''\} - 6\mathcal{L}\{y'\} + 9\mathcal{L}\{y\} = \mathcal{L}\{t\}.
\]

para \(\mathcal{L}\{t\}\) tenemos:

\[\mathcal{L}\{t\}=\frac{1}{s^2}\]

Para \(\mathcal{L}\{y''\}\), \(\mathcal{L}\{y'\}\) y \(\mathcal{L}\{y\}\):
\begin{align*}
    \mathcal{L}\{y''\} & = s^2Y(s) - sy(0) - y'(0) \\
    \mathcal{L}\{y'\}  & = sY(s) - y(0)            \\
    \mathcal{L}\{y\}   & = Y(s)
\end{align*}

Considerando la condición inicial $y(0) = 0$ y $y'(0) = 1$, tenemos:
\begin{align*}
    \mathcal{L}\{y''\} & = s^2Y(s) - s(0) - 1 \\
    \mathcal{L}\{y'\}  & = sY(s) - 0          \\
    \mathcal{L}\{y\}   & = Y(s)
\end{align*}

Sustituyendo:

\[
    s^2Y(s) - 1 - 6sY(s) + 9Y(s) = \frac{1}{s^2}
\]

\[
    Y(s)(s^2 -6s + 9) - 1 = \frac{1}{s^2}
\]

\[
    Y(s)(s^2 -6s + 9) = \frac{1}{s^2} + 1
\]

\[
    Y(s)(s-3)^2 = \frac{1}{s^2} + 1
\]

\[
    Y(s) = \frac{1}{s^2(s-3)^2} + \frac{1}{(s-3)^2}
\]


\newpage


\subsubsection{Problema 6}

\[y'' - 4y' + 4y = t^3, \hspace{.5cm} y(0) = 1, y'(0) = 0\]

Aplicamos la Transformada de Laplace a ambos lados de la ecuación:
\[
    \mathcal{L}\{y''\} - 4\mathcal{L}\{y'\} + 4\mathcal{L}\{y\} = \mathcal{L}\{t^3\}.
\]

para \(\mathcal{L}\{t^3\}\) tenemos:

\[\mathcal{L}\{t^3\} = \frac{6}{s^4}\]

Para \(\mathcal{L}\{y''\}\), \(\mathcal{L}\{y'\}\) y \(\mathcal{L}\{y\}\):
\begin{align*}
    \mathcal{L}\{y''\} & = s^2Y(s) - sy(0) - y'(0) \\
    \mathcal{L}\{y'\}  & = sY(s) - y(0)            \\
    \mathcal{L}\{y\}   & = Y(s)
\end{align*}

Considerando la condición inicial $y(0) = 1$ y $y'(0) = 0$, tenemos:
\begin{align*}
    \mathcal{L}\{y''\} & = s^2Y(s) - s(1) - 0 \\
    \mathcal{L}\{y'\}  & = sY(s) - 1          \\
    \mathcal{L}\{y\}   & = Y(s)
\end{align*}

Sustituyendo:

\[
    s^2Y(s) - s - 4(sY(s) - 1) + 4Y(s) = \frac{6}{s^4}
\]

\[
    s^2Y(s) - s - 4sY(s) - 4 + 4Y(s) = \frac{6}{s^4}
\]

\[
    Y(s)(s^2 - 4s + 4) - s = \frac{6}{s^4}
\]

\[
    Y(s)(s-2)^2 = \frac{6}{s^4} + s
\]

\[
    Y(s) = \frac{6}{s^4(s-2)^2} + \frac{1}{s(s-2)^2}
\]


\newpage


\subsubsection{Problema 7}

\[y'' - 6y' + 13y = 0, \hspace{.5cm} y(0) = 0, y'(0) = -3\]

Aplicamos la Transformada de Laplace a ambos lados de la ecuación:
\[
    \mathcal{L}\{y''\} - 6\mathcal{L}\{y'\} + 13\mathcal{L}\{y\} = \mathcal{L}\{0\}
\]

Para \(\mathcal{L}\{y''\}\), \(\mathcal{L}\{y'\}\) y \(\mathcal{L}\{y\}\):
\begin{align*}
    \mathcal{L}\{y''\} & = s^2Y(s) - sy(0) - y'(0) \\
    \mathcal{L}\{y'\}  & = sY(s) - y(0)            \\
    \mathcal{L}\{y\}   & = Y(s)
\end{align*}

Considerando la condición inicial $y(0) = 0$ y $y'(0) = 3$, tenemos:
\begin{align*}
    \mathcal{L}\{y''\} & = s^2Y(s) - s(0) - 3 \\
    \mathcal{L}\{y'\}  & = sY(s) - 0          \\
    \mathcal{L}\{y\}   & = Y(s)
\end{align*}

Sustituyendo:

\[
    s^2Y(s) - 3 - 6sY(s) + 13Y(s) = 0
\]

\[
    Y(s)(s^2 - 6s + 13) - 3 = 0
\]

\[
    Y(s) = \frac{3}{(s^2 - 6s + 13)}
\]

\newpage

\subsubsection{Problema 8}
\[2y'' + 20y' + 51y = 0, \hspace{.5cm} y(0) = 2, y'(0) = 0\]

Aplicamos la Transformada de Laplace a ambos lados de la ecuación:
\[
    2\mathcal{L}\{y''\} + 20\mathcal{L}\{y'\} + 51\mathcal{L}\{y\} = \mathcal{L}\{0\}.
\]

Para \(\mathcal{L}\{y''\}\), \(\mathcal{L}\{y'\}\) y \(\mathcal{L}\{y\}\):
\begin{align*}
    \mathcal{L}\{y''\} & = s^2Y(s) - sy(0) - y'(0) \\
    \mathcal{L}\{y'\}  & = sY(s) - y(0)            \\
    \mathcal{L}\{y\}   & = Y(s)
\end{align*}

Considerando la condición inicial $y(0) = 2$ y $y'(0) = 0$, tenemos:
\begin{align*}
    \mathcal{L}\{y''\} & = s^2Y(s) - s(2) - 0 \\
    \mathcal{L}\{y'\}  & = sY(s) - 2          \\
    \mathcal{L}\{y\}   & = Y(s)
\end{align*}

Sustituyendo:
\[
    2(s^2Y(s) - 2s) + 20(sY(s) - 2) + 51Y(s) = 0
\]

\[
    2s^2Y(s) - 4s + 20sY(s) - 40 + 51Y(s) = 0
\]

\[
    Y(s)(2s^2 + 20s + 51) - 4s - 40 = 0
\]

\[
    Y(s)(2s^2 + 20s + 51) = 4s + 40
\]

\[
    Y(s) = \frac{4s + 40}{2s^2 + 20s + 51}
\]


\newpage


\subsubsection{Problema 9}

\[y'' - y' = e^{t}cos(t), \hspace{.5cm} y(0) = 0, y'(0) = 0 \]

Aplicamos la Transformada de Laplace a ambos lados de la ecuación:
\[
    \mathcal{L}\{y''\} - \mathcal{L}\{y'\} = \mathcal{L}\{e^t\cos(t)\}.
\]

Resolviendo $\mathcal{L}\{e^t\cos(t)\}$:

Sabemos:

\[
    \mathcal{L}\{e^{at}\cos(bt)\} = \frac{s-a}{(s-a)^2 + b^2}
\]

\[
    \mathcal{L}\{e^t\cos(t)\} = \frac{s-1}{(s-1)^2 + 1^2}
\]

Para \(\mathcal{L}\{y''\}\), \(\mathcal{L}\{y'\}\) y \(\mathcal{L}\{y\}\):
\begin{align*}
    \mathcal{L}\{y''\} & = s^2Y(s) - sy(0) - y'(0) \\
    \mathcal{L}\{y'\}  & = sY(s) - y(0)
\end{align*}

Considerando la condición inicial $y(0) = 0$ y $y'(0) = 0$, tenemos:
\begin{align*}
    \mathcal{L}\{y''\} & = s^2Y(s) \\
    \mathcal{L}\{y'\}  & = sY(s)
\end{align*}

Sustituyendo:
\[
    s^2Y(s) - sY(s) = \frac{s-1}{(s-1)^2 + 1^2}
\]

\[
    Y(s)(s^2 - s) = \frac{s-1}{(s-1)^2 + 1}
\]

\[
    Y(s) = \frac{s-1}{(s^2 - s)((s-1)^2 + 1)}
\]

\[
    Y(s) = \frac{s-1}{s(s - 1)((s-1)^2 + 1)}
\]

\[
    Y(s) = \frac{1}{s((s-1)^2 + 1)}
\]

\[
    Y(s) = \frac{1}{s^3 - 2s^2 + 2s}
\]

\[
    Y(s) = \frac{1}{s(s^2 - 2s^1 + 2)}
\]

\[
    Y(s) = \frac{1}{s((s-1)^2 + 1)}
\]

\[
    \frac{1}{s((s-1)^2 + 1)} = \frac{A(s-1) + B}{(s-1)^2 + 1} + \frac{C}{s}
\]

donde $r = s-1$

\[
    \frac{Ar + B}{r^2 + 1} |^{r=s-1}
\]

\[
    \frac{Ar(r+1) + B(r+1) + C(r^2 + 1)}{(r^2 + 1)(r + 1)}
\]

\[
    \frac{Ar^2 + Ar + Br + B + Cr^2 + C}{(r^2 + 1)(r + 1)}
\]

\newpage


\subsubsection{Problema 10}
\[y'' - 2y' + 5y = 1 + t, \hspace{.5cm} y(0) = 0, y'(0) = 4\]

Aplicamos la Transformada de Laplace a ambos lados de la ecuación:
\[
    \mathcal{L}\{y''\} - 2\mathcal{L}\{y'\} + 5\mathcal{L}\{y\} = \mathcal{L}\{1 + t\}
\]

\[
    \mathcal{L}\{y''\} - 2\mathcal{L}\{y'\} + 5\mathcal{L}\{y\} = \frac{1}{s} + \frac{1}{s^2}
\]

Para \(\mathcal{L}\{y''\}\), \(\mathcal{L}\{y'\}\) y \(\mathcal{L}\{y\}\):
\begin{align*}
    \mathcal{L}\{y''\} & = s^2Y(s) - sy(0) - y'(0) \\
    \mathcal{L}\{y'\}  & = sY(s) - y(0)            \\
    \mathcal{L}\{y\}   & = Y(s)
\end{align*}

Considerando la condición inicial $y(0) = 0$ y $y'(0) = 4$, tenemos:

\begin{align*}
    \mathcal{L}\{y''\} & = s^2Y(s) - s(0) - 4 \\
    \mathcal{L}\{y'\}  & = sY(s) - (0)        \\
    \mathcal{L}\{y\}   & = Y(s)
\end{align*}

Sustituyendo:
\[
    s^2Y(s) - 4 - 2sY(s) + 5Y(s) = \frac{1}{s} + \frac{1}{s^2}
\]

\[
    Y(s)(s^2 - 2s + 5) - 4 = \frac{1}{s} + \frac{1}{s^2}
\]


\[
    Y(s) = \frac{1}{s(s^2 - 2s + 5)} + \frac{1}{s^2(s^2 - 2s + 5)} + \frac{4}{(s^2 - 2s + 5)}
\]

\newpage


\section{Funcion de impulso}

\[
    \mathcal{L}\{f(t) S(t-c)\} =
\]

\[
    S(t-c) =
\]

\end{document}